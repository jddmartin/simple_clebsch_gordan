%-*- mode: latex ;mode: visual-line -*-
\documentclass[11pt]{article}
\usepackage[left=2.5cm,top=2.0cm,right=2.5cm,bottom=3.0cm]{geometry}
\usepackage{tikz}
\usepackage{outlines}
\usepackage{braket}
\usepackage{amsmath}
\usepackage{hyperref}
\usepackage{mathtools}
\usepackage[framemethod=TikZ]{mdframed}
\mdfdefinestyle{MyFrame}{%
    linecolor=black,
    innertopmargin=-12pt}

% for the following see: http://tex.stackexchange.com/questions/11031/how-to-do-the-curvy-l-for-lagrangian-or-laplace-transforms
\newcommand{\Lagr}{\mathcal{L}}
\newcommand{\CurD}{\mathcal{D}}
\begin{document}

\begin{center}
{\large Notes on the numerical calculation of Clebsch-Gordan coefficients}
\end{center}

\noindent
Reference SN: Sakurai and Napolitano, {\em Modern Quantum Mechanics}, 2nd ed.

\begin{outline}
\1 I have implemented a simple Python function to demonstrate the calculation
of Clebsch-Gordan coefficients: 
\begin{center}\url{http://github/jddmartin/simple_clebsch_gordan} 
\end{center}
The purpose of these notes is to document the usage of some of the equations in
this code.  Boxed equations are explicitly used in the code.
\1 Notation: in this discussion we will abbreviate the Clebsch-Gordan notation of SN from $\braket{j_1j_2;m_1m_2|j_1j_2;jm}$ to
$\braket{j_1j_2;m_1,m_2|j,m}$.
\1 we begin with the general recursion relations for Clebsch-Gordan coefficients:
\begin{multline}
  C_{\pm}(j,m') \braket{j_1j_2;m_1',m_2'|j,m'\pm1}
    = \\
  C_{\pm}(j_1,m'_1\mp1) \braket{j_1j_2;m'_1\mp1,m'_2|j,m'}
  +
  C_{\pm}(j_2,m'_2\mp1) \braket{j_1j_2;m'_1,m'_2\mp1|j,m'}
\label{eq:general_recursion}
\end{multline}
where 
\begin{equation}
\boxed{
C_{\pm}(j,m) = \sqrt{j (j+1) - m (m\pm1)}.
}
\end{equation}

For the derivation of Eq.~\ref{eq:general_recursion}, see pages 224-225 of SN culminating in SN3.8.49.

\1 for evaluation of the general case $\braket{j_1j_2;m_1,m_2|j,m}$, when $m \ne j$, we will use the lower sign version of Eq.~\ref{eq:general_recursion} to determine a Clebsch-Gordan coefficient $\braket{j_1j_2;m_1,m_2|j,m}$ in terms of the coefficents: 
\begin{center}
$\braket{j_1j_2;m_1+1,m_2|j,m+1}$
and
$\braket{j_1j_2;m_1\,m_2+1|j,m+1}$.  
\end{center}
The relevant equation is determined by substitution of 
$m'=m+1$,
$m_1'=m_1 $ and 
$m_2'=m_2 $
into the lower sign recursion relation of Eq.~\ref{eq:general_recursion}, to give:
\begin{mdframed}[style=MyFrame]
\begin{multline}
\braket{j_1j_2;m_1,m_2|j,m} = \\
\frac{1}{C_-(j,m+1)}
[
C_-(j_1,m_1+1) \braket{j_1j_2;m_1+1,m_2|j,m+1}
+ \\
C_-(j_2,m_2+1) \braket{j_1j_2;m_1,m_2+1|j,m+1}
].
\end{multline} \end{mdframed}
\1 when we have to evaluate the Clebsch-Gordan coefficients 
$\braket{j_1j_2;m_1,j-m_1|j,j}$ we will use the upper sign recursion relation
of Eq.~\ref{eq:general_recursion} to determine this coefficient in terms of the \\
$\braket{j_1j_2;m_1+1,j-m_1-1|j,j}$ coefficient.
\1 the relevant formula can be obtained from the general recursion relation
by substitution of $m_1'=m_1+1$ and $m_2'=j-m_1$ into the upper sign version of Eq.~\ref{eq:general_recursion} and rearrangement:
\begin{equation}
\boxed{
\braket{j_1j_2;m_1,j-m_1|j,j} 
= \frac{- \: C_+(j_2,j-m_1-1)}{C_+(j_1,m_1)} \braket{j_1j_2;m_1+1,j-m_1-1|j,j}.
}
\label{eq:downwards}
\end{equation}

\1 we must be able to calculate the special Clebsch-Gordan coefficient 
$\braket{j_1j_2;j_1,j-j_1|j,j}$
(the termination of the recursion relation of Eq.~\ref{eq:downwards}).

This coefficient is defined by convention to be both positive and real.  To determine its magnitude, we use the orthonormality constraints of the Clebsch-Gordan coefficients (see SN3.8.42):
\begin{equation}
\sum_{m_1,m_2} \braket{j_1j_2;m_1,m_2|j,m}
\braket{j_1j_2;m_1,m_2,j',m'} = \delta_{j,j'} \delta_{m,m'}
\end{equation}
specializing to:
\begin{equation}
\sum_{m_1=j_1, j_1-1, ...} |\braket{j_1j_2;m_1,j-m_1|j,j}|^2 = 1.
\end{equation}
Rearranging gives:
\begin{multline}
\frac{1}{|\braket{j_1j_2;j_1,j-j_1|j,j}|^2}
  = \\ 1 + \frac{1}{|\braket{j_1j_2;j_1,j-j_1|j,j}|^2}
\sum_{m_1=j_1-1, j_1-2, ...}
|\braket{j_1j_2;m_1,j-m_1|j,j}|^2
\end{multline}
This equation can be written in a slightly different form suitable for use of a specialized form of the general recursion relationship (Eq.~\ref{eq:general_recursion}):
\begin{equation}
\frac{1}{|\braket{j_1j_2;j_1,j-j_1|j,j}|^2}
  = 1 + f(j_1) \times [1 + f(j_1-1) \times [ 1 + f(j_1-2) \times [  ... ]]]
\label{eq:recursive_sum}
\end{equation}
where
\begin{equation}
f(m_1) \equiv \left| 
  \frac{
  \braket{j_1j_2;m_1-1,j-(m_1-1)|j,j}
  }{
  \braket{j_1j_2;m_1,j-m_1|j,j}}
  \right|^2 .
\end{equation}
We can obtain an expression for $f(m_1)$ from the upper-sign version of the general recursion relationship (Eq.~\ref{eq:general_recursion}). Substitution of 
$m_1'= m_1$ and $m_2'= j-m_1+1$ and rearrangment gives:
\begin{equation}
\boxed{
f(m_1)
 = \left|
\frac{C_+(j_2,j-m_1)}{C_+(j_1,m_1-1)}
   \right|^2 .
}
\label{eq:for_f}
\end{equation}

\1 note Eq.~\ref{eq:recursive_sum} may be evaluated using recursion:
\begin{equation}
\boxed{
s(m_1) = 1 + f(m_1) \times s(m_1-1).
}
\end{equation}
The recursion terminates when the Clebsch-Gordan
coefficient in the numerator of Eq.~\ref{eq:for_f}
is known to be zero.  i.e. the magnitude of either one of the uncoupled magnetic quantum numbers are larger than allowed by $j_1$ and $j_2$.
\1 finally, the required Clebsch-Gordan coefficient is (through rearrangment of Eq.~\ref{eq:recursive_sum}):
\begin{equation}
\boxed{
\braket{j_1j_2;j_1,j-j_1|j,j} = \frac{1}{\sqrt{s(j_1)}},
}
\end{equation}
making use of the convention that this coefficient is real and positive.
\end{outline}

Comments are welcome. - JDDM.

\end{document}

